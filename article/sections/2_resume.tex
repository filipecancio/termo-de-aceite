\section{Resumo do trabalho}

Existem diversas plataformas para digitar um texto acadêmico e junto com elas diversos tipos de normas para revistas e congressos acadêmicos. Sabendo o desafio que é para o autor de um artigo passar suas ideias para o papel, e de forma contínua o ajustar à normas e padrões estabelecidos, algumas ferramentas foram desenvolvidas para facilitar a escrita de artigos.

É o caso do LaTex, uma espécie de sistema de documentos desenvolvido em 1984 com a linguagem de marcação TEX que permite que um simples texto possa ser transpilado para uma quantidade infinita de documentos, cada um com uma norma diferente.

O LaTex foi desenvolvido em meados dos anos 80 e hoje está presente aplicativos como MikTex e sites como Overleaf com algumas limitações ou funcionalidades que precisam ser pagas para utilizar.

Em 2023 no Instituto Federal de Educação Ciência e Tecnologia - Bahia do campus de Vitória da Conquista, foi aprovado pela coordenação de Sistemas de Informação a utilização do formato da Sociedade Brasileira de Computação (SBC) como norma para a escrita do trabalho de conclusão de curso. Assim os alunos podem pegar os templates disponíveis no site da SBC como base para a escrita do seu trabalho, entre eles um template em LaTex.

Com base nessas informações, o artigo presente aborda o desenvolvimento de um template com o formato de repositório do GitHub utilizando como norma as regas da SBC. Atraves dos repositórios "accept-term-template" e "sbc-template" o usuário poderá criar o aceite de termos e o artigo com funcionalidades que excedem serviços como o Overleaf de forma gratuita e acessível.

Nesses repositórios é possível editar o artigo de forma online em qualquer dispositivo (também disponível offline) e pegar a ultima versão do pdf de forma automatizada. As versões anteriores também ficarão salvas e disponíveis em formato de discursão permitindo completa integração aluno-orientador, usando discussões e cronogramas em GANTT anexada à versão discutida do pdf.

Veremos nesse artigo todo o processo de criação desses repositórios envolvendo o uso do LaTex para a criação de um template simples para usuários iniciantes, o uso de ferramentas de DevOps para criar um ambiente completo com dependencias e extensões Latex, controle de versionamento e editor de texto (Visual Studio Code), e por fim automação dos serviços e ferramentas de gestão pelo github.